%  Overzichtsbladzijde met samenvatting

\newpage

{
\setlength{\baselineskip}{14pt}
\setlength{\parindent}{0pt}
\setlength{\parskip}{8pt}

\begin{center}

\noindent \textbf{\huge
Internet of Things: Integratie in Drupal
}

door 

Kobe WRIGHT\\
Stef DE WAELE

Masterproef voorgedragen tot het behalen van het diploma van\\
Master in de industri\"{e}le wetenschappen: informatica

Academiejaar 2012--2013

Promotoren: dr.~ir.~P.~SIMOENS, prof.~dr.~ir.~I.~MOERMAN, dr.~ir.~J.~HOEBEKE\\
Begeleiders: ing.~J.~ROSSEY, ir.~F.~VAN DEN ABEELE\\

Geassocieerde Faculteit Toegepaste Ingenieurswetenschappen\\
Hogeschool Gent

Vakgroep Informatica

\end{center}

\section*{Samenvatting}

% TODO: samenvatting

Het Internet of Things (IoT), waarbij verscheidene soorten apparaten met het internet worden verbonden, wordt steeds belangrijker. Het biedt de mogelijkheid deze apparaten van op afstand te observeren en besturen op een gemakkelijke en dynamische manier.\\
Het succesvolle en wijd verspreide HTTP-protocol is echter te zwaar, het veroorzaakt te veel overhead om gebruikt te kunnen worden in sensornetwerken, waar de bandbreedte beperkt is. Bovendien wordt de nodige energie voor sensoren vaak geleverd door een batterij, dus is het belangrijk energieconsumptie te beperken. In ruil voor de betrouwbaarheid en robuustheid van het HTTP-protocol biedt het CoAP-protocol minder betrouwbare communicatie, maar het veroorzaakt slechts een minimale overhead en biedt een sensor de mogelijkheid op eigen initatief data te sturen naar clients die ge\"{i}nteresseerd zijn.\\
Het doel van deze masterproef bestaat erin een Drupal-module te ontwikkelen die het mogelijk maakt op een gebruiksvriendelijke en dynamische manier sensor data op te vragen en eventueel sensoren te configureren. Bovendien moeten sensoren in het subnetwerk gevonden worden a.d.h.v. een resource directory. Dit alles moet gebeuren door enkel gebruik te maken van native CoAP-communicatie. 


\section*{Trefwoorden}

% TODO: trefwoorden

Internet of things, Drupal, CoAP, sensornetwerk

}

\newpage % strikt noodzakelijk om een header op deze blz. te vermijden
