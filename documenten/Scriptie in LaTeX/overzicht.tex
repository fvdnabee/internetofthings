%  Overzichtsbladzijde met samenvatting

\newpage

{
\setlength{\baselineskip}{14pt}
\setlength{\parindent}{0pt}
\setlength{\parskip}{8pt}

\begin{center}

\noindent \textbf{\huge
Internet of Things: Integratie in Drupal
}

door 

Kobe WRIGHT\\
Stef DE WAELE

Masterproef voorgedragen tot het behalen van het diploma van\\
Master in de industri\"{e}le wetenschappen: informatica

Academiejaar 2012--2013

Promotoren: dr.~ir.~J.~HOEBEKE, prof.~dr.~ir.~I.~MOERMAN, dr.~ir.~P.~SIMOENS\\
Begeleiders: ing.~J.~ROSSEY, ir.~F.~VAN DEN ABEELE\\

Geassocieerde Faculteit Toegepaste Ingenieurswetenschappen\\
Hogeschool Gent

Vakgroep Informatica

\end{center}

\section*{Samenvatting}

% TODO: samenvatting

Het Internet of Things wordt steeds belangrijker. Bij dit concept staat het idee dat verscheidene soorten apparaten rechtstreeks of onrechtstreeks met het internet verbonden worden centraal. Het biedt de mogelijkheid deze apparaten van op afstand te observeren en te besturen op een gemakkelijke en dynamische manier. Deze apparaten zijn in werkelijkheid vaak beperkt in mate van energieverbruik en geheugen. Vaak is het netwerk waarmee ze verbonden zijn eveneens beperkt waardoor beperkte bandbreedte en onbetrouwbare verbindingen niet zeldzaam zijn.

Het doel van deze masterproef bestaat erin een Drupal-module te ontwikkelen die het mogelijk maakt op een gebruiksvriendelijke en dynamische manier sensordata op te vragen en sensoren te configureren. Hierbij moet rekening gehouden worden met de beperkte mogelijkheden van zowel het netwerk als van de aangesloten apparaten. Alsook moet de module ruimte laten voor uitbreidingen en moeten de recente mogelijkheden van Drupal 7 benut worden met een oog op de aangeboden functionaliteit van Drupal 8, enige commentaar in code mag dan ook niet ontbreken.

Om de communicatie tussen de devices over het netwerk te waarborgen is HTTP te zwaar. Daarom wordt er gebruik gemaakt van het minder zware CoAP. In ruil voor de betrouwbaarheid en robuustheid van HTTP biedt CoAP een minimale overhead, een hoge mate van simpliciteit en features die zich niet beperken tot de observe-functionaliteit en blockwise transfer.

Als resultaat bieden we twee Drupal-modules aan. De CoAP library en de CoAP Resource module. De CoAP library verzorgt alle communicatie tussen clients en servers op een netwerk d.m.v. native CoAP. Deze module heeft op zich weinig te bieden aan de doorsnee Drupal-gebruiker en heeft enkel als doel een CoAP API aan te bieden aan Drupal-ontwikkelaars. Deze stand-alone module kan ook gebruikt worden in andere projecten waar een CoAP-implementatie in Drupal vereist is. De CoAP Resource module maakt gebruik van de CoAP library en biedt de gebruiksvriendelijke interface aan waarmee Drupal-gebruikers sensoren kunnen beheren.


%Het Internet of Things (IoT), waarbij verscheidene soorten apparaten met het internet worden verbonden, wordt steeds belangrijker. Het biedt de mogelijkheid deze apparaten van op afstand te observeren en besturen op een gemakkelijke en dynamische manier.\\
%Het HTTP-protocol  veroorzaakt te veel overhead om gebruikt te kunnen worden in sensornetwerken, waar de bandbreedte beperkt is. Bovendien wordt de nodige energie voor sensoren vaak geleverd door een batterij, dus is het belangrijk energieconsumptie te beperken. In ruil voor de betrouwbaarheid en robuustheid van het HTTP-protocol biedt het CoAP-protocol minder betrouwbare communicatie, maar het veroorzaakt slechts een minimale overhead en biedt een sensor de mogelijkheid op eigen initatief data te sturen naar clients die ge\"{i}nteresseerd zijn.\\
%Het doel van deze masterproef bestaat erin een Drupal-module te ontwikkelen die het mogelijk maakt op een gebruiksvriendelijke en dynamische manier sensordata op te vragen en eventueel sensoren te configureren. Bovendien moeten sensoren in het subnetwerk gevonden worden a.d.h.v. een resource directory. Dit alles moet gebeuren door enkel gebruik te maken van native CoAP-communicatie. 


\section*{Trefwoorden}

% TODO: trefwoorden

Internet of things, Drupal, CoAP, sensornetwerk

}

\newpage % strikt noodzakelijk om een header op deze blz. te vermijden
