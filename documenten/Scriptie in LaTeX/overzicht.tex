%  Overzichtsbladzijde met samenvatting

\newpage

{
\setlength{\baselineskip}{14pt}
\setlength{\parindent}{0pt}
\setlength{\parskip}{8pt}

\begin{center}

\noindent \textbf{\huge
Internet of Things: Integratie in Drupal
}

door 

Kobe WRIGHT\\
Stef DE WAELE

Masterproef voorgedragen tot het behalen van het diploma van\\
Master in de industri\"{e}le wetenschappen: informatica

Academiejaar 2012--2013

Promotoren: dr.~ir.~J.~HOEBEKE, prof.~dr.~ir.~I.~MOERMAN, dr.~ir.~P.~SIMOENS\\
Begeleiders: ing.~J.~ROSSEY, ir.~F.~VAN DEN ABEELE\\

Geassocieerde Faculteit Toegepaste Ingenieurswetenschappen\\
Hogeschool Gent

Vakgroep Informatica

\end{center}

\section*{Samenvatting}

% TODO: samenvatting

Het Internet of Things wordt steeds belangrijker. Bij dit concept staat het idee dat verscheidene soorten apparaten rechtstreeks of onrechtstreeks met het internet verbonden worden centraal. Het biedt de mogelijkheid deze apparaten van op afstand te observeren en te besturen op een gemakkelijke en dynamische manier. Deze apparaten zijn in werkelijkheid vaak beperkt in mate van energieverbruik en geheugen. Vaak is het netwerk waarmee ze verbonden zijn eveneens beperkt waardoor beperkte bandbreedte en onbetrouwbare verbindingen niet zeldzaam zijn.

Het doel van deze masterproef bestaat erin een Drupal-module te ontwikkelen die het mogelijk maakt op een gebruiksvriendelijke en dynamische manier sensordata op te vragen en sensoren te configureren. Hierbij moet rekening gehouden worden met de beperkte mogelijkheden van zowel het netwerk als van de aangesloten apparaten. Ook moet de module ruimte laten voor uitbreidingen en moeten de recente mogelijkheden van Drupal 7 benut worden met een oog op de aangeboden functionaliteit van Drupal 8. Enige commentaar in code mag dan ook niet ontbreken.

Om de communicatie tussen de devices over het netwerk te waarborgen is HTTP te zwaar. Daarom wordt er gebruik gemaakt van het minder zware CoAP. In ruil voor de betrouwbaarheid en robuustheid van HTTP biedt CoAP een minimale overhead, een hoge mate van simpliciteit en features die zich niet beperken tot de observe-functionaliteit en blockwise transfer.

Als resultaat bieden we twee Drupal-modules aan. De CoAP library- en de CoAP Sensor-module. De CoAP library verzorgt alle communicatie tussen clients en servers op een netwerk door middel van native CoAP. Deze module heeft op zich weinig te bieden aan de doorsnee Drupal-gebruiker en heeft enkel als doel een CoAP API aan te bieden aan Drupal-ontwikkelaars. Deze stand-alone module kan ook gebruikt worden in andere projecten waar een CoAP-implementatie in Drupal vereist is. De CoAP Sensor-module maakt gebruik van de CoAP library en biedt de gebruiksvriendelijke interface aan waarmee Drupal-gebruikers sensoren kunnen beheren.

\section*{Abstract}
The Internet of Things is becoming more important every day. The Central idea in this concept is that multiple kinds of devices are directly or indirectly connected to the internet. It offers the possibility to observe or configure these devices in a user-friendly and dynamic way. These devices are more often than not constrained in both energy supply and memory. Most of the time the network on which they are located is also constrained. Which results in low bandwidth and unreliable connections. 

The aim of this thesis is to develop a Drupal module which allows a user-friendly and dynamic way to collect sensor data and to configure sensors. It must be taken into account that the devices and the network which they are connected to, are constrained. The module also needs to leave room for expansions and needs to take advantage from the recent possibilities from Drupal 7 while not neglecting the functionalities of Drupal 8. Therefore comments in the code should not be absent. 

HTTP is too heavy to make the communication between the devices possible. Because of this, CoAP is used instead, which is a lightweight protocol. CoAP exchanges the reliability and robustness from HTTP for a minimal overhead, a simplified design and features which are not limited to the observe functionality and blockwise transfer.

As a result we present two Drupal modules. The CoAP library- and the CoAP Sensor module. The CoAP library takes care of all the communication between clients and servers on a network, using native CoAP. This module doesn’t offer much to the average Drupal user and its only purpose is to offer a CoAP API to Drupal developers. This stand-alone module can also be used in other projects where a CoAP implementation is required. The CoAP Sensor module uses the CoAP library to offer the user-friendly interface which Drupal users can use to maintain sensors.
\section*{Trefwoorden}

% TODO: trefwoorden

Internet of things, Drupal, CoAP, sensornetwerk

}

\newpage % strikt noodzakelijk om een header op deze blz. te vermijden
