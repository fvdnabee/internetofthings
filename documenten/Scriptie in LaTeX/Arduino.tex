\chapter{Arduino}

Het device zelf heeft ook een rol als proxy om de eindgebruiker het idee te geven dat de sensoren rechtstreeks aanspreekbaar zijn, terwijl eigenlijk het embedded device een vorm van controle invoert op het gebruik van kostbare bandbreedte.
Een device kan namelijk meerdere sensoren omvatten (temperatuur, vochtigheid, etc). Bij het gebruik van deze module moeten sensoren automatisch gedetecteerd worden, wanneer deze fysiek op het embedded device worden aangesloten. Om de automatische detectie mogelijk te maken wordt gebruik gemaakt van een soort publieke directory, die de sensoren bevat onder vorm van een lijst. Wanneer een sensor aangesloten wordt op het device, moet de Drupal-module de eindgebruiker hiervan op de hoogte stellen en moet het mogelijk zijn om de sensor te configureren naar eigen wensen.